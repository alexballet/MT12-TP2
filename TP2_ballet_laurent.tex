%Information relative au document
% !TEX encoding = MacOSRoman
\documentclass[a4paper,12pt]{report}


%Paquets
\usepackage[T1]{fontenc}
\usepackage[applemac]{inputenc}
\usepackage{xspace}
\usepackage[frenchb]{babel}
\usepackage{textcomp}
\usepackage{lmodern}
\usepackage{setspace}
\onehalfspacing
\usepackage{graphicx}
\usepackage{amsmath}
\usepackage{amsfonts}
%Pour faire un tableau de d\'eriv\'e
\usepackage{tikz,tkz-tab}
\usepackage{float}
\usepackage{here}
\usepackage[left=2cm,right=2cm,top=3cm,bottom=3cm]{geometry}
\usepackage{listings}
%Pour inclure du code
\lstset{numbers=left, numberstyle=\tiny, stepnumber=5, numbersep=5pt, language=Scilab, basicstyle=\small,frame=leftline,captionpos=b,linewidth=175mm,breaklines=true, commentstyle=\color{green},stringstyle=\color{red},identifierstyle=\ttfamily,keywordstyle=\color{blue}}



%D\'ebut du rapport
\begin{document}

%Page de garde
%\title{Rapport TP \no1 MT12}
%\author{Alexandre BALLET et Simon LAURENT}
%\date{Printemps 2016}

\thispagestyle{empty}

\includegraphics[scale=0.06]{logo_utc.png}

{\large

\vspace*{1cm}

\begin{center}

{\bf Rapport de MT12 : Techniques math\'ematiques de l'ing\'enieur \\ UNIVERSIT\'E DE TECHNOLOGIE DE COMPI\`EGNE}

\vspace*{1 cm}

\vspace*{1cm}

Printemps 2016

\vspace*{1cm}

\vspace*{1cm}
{\Large {\bf Alexandre BALLET et Simon LAURENT}}
\vspace*{2cm}

\vspace*{2 cm}
\end{center}

\flushleft{Sujet du rapport:\ \\
{\Large {\bf Transform\'ee de Fourier discr\`ete}}}

\vspace*{1 cm}
\flushleft{D\'epartement des \'etudiants :\ \\
{\Large {\bf  G\'enie Informatique}}}\\

\vspace*{1 cm}
\flushleft{Professeur :\ \\
{\Large {\bf M. Djalil KATEB}}}\\
}

\newpage
\null
\newpage
%Table des mati\`eres
\tableofcontents


%Questions 2.0 et 2.1
\chapter{Définition de la DFT}

	%Question 2.0
	\section{D\'efinition}

	%Question 2.1
	\section{DFT d'une fonction}

	Soit f d\'efinie sur une p\'eriode T (p\'eriodique de p\'eriode T) et soit $y_{k}$ une suite d'\'echantillons de $f$ en N points uniform\'ement r\'epartis sur une p\'eriode. La DFT de $f$ d'ordre $N$ est l'application qui associe \`a la suite $(y_{k})_{k=0,...,N-1}$ la suite constitu\'ee de la DFT appliqu\'ee au vecteur $y = (y_{k})_{k=0,...,N-1}$. On note $(DFT[f](k))_{0,...,N-1}$ la suite obtenue et $iDFT$ sa r\'eciproque.

	Nous cherchons à montrer que l'on peut approcher le coefficient de Fourier $c_{n}$ \`a partir de l'approximation de l'int\'egrale la d\'effinissant par une somme de Riemann.

	D'ap\`es la d\'efinition de $c_{n}$ on a : \[c_{n}=\frac{1}{T}\int_{0}^{T}f(t)e^{-2i\pi \frac{n}{T}t}dt\]

	Si nous discr\`etisons $f(t)$ en divisant une p\'eriode $T$ en $N$ segments, on obtient :

	\begin{eqnarray*}
		c_{n}&=&\frac{1}{T}\int_{0}^{T}f(t)e^{-2i\pi \frac{n}{T}t}dt\\
		&=&\frac{1}{T}\sum_{k=0}^{N-1}f(t_{k})e^{-2i\pi \frac{n}{T}k\frac{T}{N}}\frac{T}{N}\\
		&=&\frac{1}{N}\sum_{k=0}^{N-1}y_{k}e^{-2i\pi n\frac{k}{N}}\\
		&=&\frac{1}{N}\sum_{p=0}^{N-1}y_{p}e^{-2i\pi n\frac{p}{N}}\\
		&=&\frac{1}{N}\sum_{p=0}^{N-1}y_{p}w^{np}\\
	\end{eqnarray*}

	D'o\`u \[c_{k}=\frac{1}{N}\sum_{p=0}^{N-1}y_{p}w^{pk}\]

	Nous pouvons en d\'eduire que la DFT d'une suite $y_{k}$ est \'equivalent au coefficient de Fourier $c_{k}$ de cette s\'erie.
	
%Question 2.2
\chapter{Une famille de fonctions tests}

%Questions 2.3 et 2.4
\chapter{Applications de la DFT}

	%Question 2.3
	\section{Approximation des coefficients de Fourier}

	%Question 2.4
	\section{R\'esolution fr\'equentielle de signaux}

		\subsection{Mise en jambes}

		\subsection{Exercice}

%Question 2.5
\chapter{DFT et convolution}

%Question 2.6
\chapter{Ph\'enom\`ene de Gibbs}

%Question 2.8
\chapter{Effet du fen\^etrage}

%Table des figures
\listoffigures

\end{document}